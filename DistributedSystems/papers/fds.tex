Flat Datacenter Storage

Abstract
Flat Datacenter Storage (FDS) is a high-performance, fault-tolerant,
large-scale, locality-oblivious blob store.
Using a novel combination of full bisection bandwidth networks, data and
metadata striping, and flow control, FDS multiplexes an application’s
large-scale I/O across the available throughput and latency budget of every disk
in a cluster. 
FDS therefore makes many optimizations around data locality unnecessary. Disks
also communicate with each other at their full bandwidth, making recovery from disk failures extremely fast. FDS is designed for datacenter scale, fully distributing metadata
operations that might otherwise become a bottleneck.
FDS applications achieve single-process read and
write performance of more than 2 GB/s. We measure recovery of 92 GB data lost to disk failure in 6.2 s and recovery from a total machine failure with 655 GB of data
in 33.7 s. Application performance is also high: we describe our FDS-based sort application which set the 2012
world record for disk-to-disk sorting.

1

Introduction

A shared and centralized model of storage is one of
simplicity. Consider a centralized file server in a small
computer science department. Data stored by any computer can be retrieved by any other. This conceptual
simplicity makes it easy to use: computation can happen on any computer, even in parallel, without regard to
first putting data in the right place. As a result, applications are much less complex than they would be without
a shared filesystem.
At the scale of large, big-data clusters that routinely
exceed thousands of computers, this “flat” model of storage is still highly desirable. While some blob storage systems such as Amazon S3 [10] provide one, they
come with a significant performance penalty because
networks at datacenter scales have historically been oversubscribed. Individual machines were typically attached
in a tree topology [12]; for cost efficiency, links near the
root had significantly less capacity than the aggregate capacity below them. Core oversubscription ratios of hundreds to one were common, which meant communication
∗ Work

completed during a Microsoft Research internship.

USENIX Association  

was fast within a rack, slow off-rack, and worse still for
nodes whose nearest ancestor was the root.
The datacenter bandwidth shortage has had unfortunate and far-reaching consequences in systems where
performance is paramount. Software developers accustomed to treating the network as an abstraction are
forced to think in terms of “rack locality.” New programming models (e.g., MapReduce [13], Hadoop [1],
and Dryad [19]) emerged to help exploit locality, preferentially moving computation to data rather than viceversa. They effectively expose a cluster’s aggregate disk
throughput for tasks with high reduction factors (searching for a rare string), but many important computations
(sort, distributed join, matrix operations) fundamentally
require data movement and are still not well served by
systems whose performance is locality-dependent. Software must also be expressed in a data-parallel style,
which is unnatural for many tasks.
Counterintuitively, locality constraints can sometimes
even hinder efficient resource utilization. One example is
stragglers: if data is singly replicated, a single unexpectedly slow machine can preclude an entire job’s timely
completion even while most of the resources are idle.
The preference for local disks also serves as a barrier
to quickly retasking nodes: since a CPU is only useful for processing the data resident there, retasking requires expensive data movement. If the resident data is
not needed, the CPUs may not be usable. In addition, because a node has a fixed ratio of CPUs to disks, tasks that
run there will nearly always leave either CPUs or disks
partially idle, depending on the resource ratio required
by the task.
The root of this cascade of consequences was the locality constraint, itself rooted in the datacenter bandwidth shortage. When bandwidth was scarce, these sacrifices were necessary to achieve the best performance.
However, recently developed CLOS networks [16, 15,
24]—large numbers of small commodity switches with
redundant interconnections—have made it economical to
build non-oversubscribed full bisection bandwidth networks at the scale of a datacenter for the first time. Flat
Datacenter Storage (FDS) is a datacenter storage system
designed from first principles under the formerly unrealistic assumption that datacenter bandwidth is abundant.

10th USENIX Symposium on Operating Systems Design and Implementation (OSDI ’12)  1


2

Unconventionally for a system of this scale, FDS returns to the flat storage model: all compute nodes can access all storage with equal throughput. Even when computation is co-located with storage, all storage is treated
as remote; in FDS, there are no “local” disks. By spreading data over disks uniformly at a relatively fine grain
(§2.2), FDS statistically multiplexes workloads over all
of the disks in a cluster. FDS effectively eliminates the
need to imbue locality constraints into storage systems,
schedulers, or programming models.

Design Overview

FDS’ main goal is to expose all of a cluster’s disk
bandwidth to applications. Blobs are divided into tracts
(§2.1), parallelizing both storage of data (§2.2) and handling of metadata (§2.3) across all disks. We exploit our
locality-oblivious storage to dynamically assign work to
workers, preventing stragglers (§2.4). FDS provides the
best of both worlds: a scale-out system with aggregate
I/O throughput equal to systems that exploit local disks
combined with the conceptual simplicity and flexibility
of a logically centralized storage array.
For simplicity we will first describe the system without regard to fault tolerance. §3 will describe replication,
failure recovery, and cluster growth. In addition, this section assumes the network core is never congested; our
network is described in §4.

Despite the conceptual simplicity afforded by the flat
storage model, FDS achieves cluster-wide I/O performance on par with systems that exploit locality. Singleprocess read and write performance exceeds 2 GB/s
(§5.2), and FDS dramatically accelerates data movement
workloads. For example, our sorting application beat a
world record for disk-to-disk sort performance (§6.1) by
a factor of 2.8 while using about 1/5 as many disks. It
is the first system in the competition’s history to do so
without exploiting locality.

2.1

Blobs and Tracts

In FDS, data is logically stored in blobs. A blob is a
byte sequence named with a 128-bit GUID. The GUID
can either be selected by the application or assigned randomly by the system. Blobs can be any length up to the
system’s storage capacity. Reads from and writes to a
blob are done in units called tracts. Each tract within a
blob is numbered sequentially starting from 0. Blobs and
tracts are mutable; nothing prevents a client from overwriting a previously-written tract with new data.
Tracts are sized such that random and sequential access achieves nearly the same throughput. In our cluster,
tracts are 8 MB (§5.1). The tract size is set when the cluster is created based upon cluster hardware. For example,
if flash were used instead of disks, the tract size could be
made far smaller (e.g., 64 kB).
Every disk is managed by a process called a tractserver that services read and write requests that arrive
over the network from clients. Tractservers do not use a
file system. Instead, they lay out tracts directly to disk by
using the raw disk interface. Since there are only about
106 tracts per disk (for a 1 TB disk), all tracts’ metadata
is cached in memory, eliminating many disk accesses.
Tractservers and their network protocol are not exposed directly to FDS applications. Instead, these details
are hidden in a client library with a narrow and straightforward interface. Figure 1 shows a simplified version of
it; some parameters and return values have been elided.
In addition to the listed parameters, each function takes a
callback function and an associated context pointer. All
calls in FDS are non-blocking; the library invokes the
application’s callback when the operation completes.
The application’s callback functions must be reentrant; they are called from the library’s threadpool and
may overlap. Tract reads are not guaranteed to arrive in
order of issue. Writes are not guaranteed to be committed in order of issue. Applications with ordering require-

A consequence of our design is that disk-to-disk bandwidth is also extremely high, facilitating fast recovery
from disk and machine failures. In our 1,000 disk cluster, FDS recovers 92 GB lost from a failed disk in 6.2
seconds. Recovery of 655 GB lost from a failed 7-disk
machine takes 33.7 seconds (§5.3).
FDS is more efficient for many workloads because every job can use the cluster’s I/O bandwidth and CPU
resources in exactly the ratio required. FDS therefore
moves away from conflating high performance with dataparallel programming. Further, since data locality is immaterial to compute nodes, they are easily and quickly
retasked. This can even be done at a fine grain within a
single task; as demonstrated in §2.4, dynamic work allocation retasks at the granularity of individual data reads
to dramatically reduce the effect of stragglers.
Much past research has been directed towards solving
the individual problems brought about by the need for
locality. FDS, in contrast, is a clean redesign from which
many of these solutions fall out naturally. Our goal is
to move datacenters back to a flat storage model so that
these benefits may be widely realized.
The remainder of this paper is organized as follows.
Section 2 gives an overview of the design of FDS. Section 3 describes FDS’ strategy for replication and failure
recovery. Section 4 explores our network in greater detail
and describes our novel congestion avoidance strategy
needed in full bisection bandwidth networks. Section 5
presents microbenchmarks. Section 6 reviews how FDS
can speed up real workloads, including sort and serving
a web index. We review related work in Section 7 and
conclude in Section 8.
2

2  10th USENIX Symposium on Operating Systems Design and Implementation (OSDI ’12)  

USENIX Association


Getting access to a blob
CreateBlob(UINT128 blobGuid)
OpenBlob(UINT128 blobGuid)
CloseBlob(UINT128 blobGuid)
DeleteBlob(UINT128 blobGuid)
Interacting with a blob
GetBlobSize()
ExtendBlobSize(UINT64 numberOfTracts)
WriteTract(UINT64 tractNumber, BYTE *buf)
ReadTract(UINT64 tractNumber, BYTE *buf)
GetSimultaneousLimit()

system, each TLT entry contains the address of a single tractserver. With k-way replication, each entry has k
tractservers; see §3.3.
To read or write tract number i from a blob with GUID
g, a client first selects an entry in the TLT by computing
an index into it called the tract locator, designed to both
be deterministic and produce uniform disk utilization:
Tract Locator = (Hash(g) + i) mod TLT Length
Hashing the GUID “randomizes” each blob’s starting
point in the table, ensuring clients better exploit the available parallelism whether or not the GUIDs themselves
are assigned randomly. FDS uses SHA-1 for this hash.
Adding the tract number outside the hash ensures that
large blobs use all entries in the TLT uniformly. An early
(discarded) locator equation used Hash(g + i). This effectively selected a TLT entry independently at random
for each tract, producing a binomial rather than a uniform distribution of tracts on tractservers. As the number
of tractservers increased, so did the occupancy deviation
between the most-filled and least-filled disk. In the rejected design, writing 1 TB of 8 MB tracts to 1,000 tractservers was expected to write between 92 and 161 tracts
to each (µ = 125; σ = 11.2). The one tractserver with
29% more data than average was an unwanted straggler.
Once clients find the proper tractserver address in the
TLT, they send read and write requests containing the
blob GUID, tract number, and (in the case of writes) the
data payload. Readers and writers rendezvous because
tractserver lookup is deterministic: as long as a reader
has the same TLT the writer had when it wrote a tract, a
reader’s TLT lookup will point to the same tractserver.
In a single-replicated system, the TLT is constructed
by concatenating m random permutations of the tractserver list. Using only a single permutation can lead to
unwanted client convoys. Sequential reads from a blob
use TLT entries sequentially, so clients that bunch up in
the queue of a slow tractserver will move in lockstep
through the TLT, overloading some tractservers while
many others are idle. Setting m > 1 ensures that after
being delayed in a slow queue, clients will fan out to m
other tractservers for their next operation. Our system
uses m = 20, but we have not tested the system’s sensitivity to this parameter.
In the case of non-uniform disk speeds, the TLT is
weighted so that different tractservers appear in proportion to the measured speed of the disk.
To be clear, the TLT does not contain complete information about the location of individual tracts in the system. It is not modified by tract reads and writes. The only
way to determine if a tract exists is to contact the tractserver that would be responsible for the tract if it does
exist. Since the TLT changes only in response to cluster reconfiguration or failures it can be cached by clients

Figure 1: FDS API
ments are responsible for issuing operations after previous acknowledgments have been received, rather than
concurrently. FDS guarantees atomicity: a write is either
committed or failed completely.
The non-blocking API helps applications achieve
good performance. By spreading a blob’s tracts over
many tractservers (§2.2) and issuing many requests in
parallel, many tractservers can begin reading tracts off
disk and transferring them back to the client simultaneously. In addition, deep read-ahead allows a tract
to be read off disk into the tractserver’s cache while
the previous one is transferred over the network. The
FDS API GetSimultaneousLimit() tells the application how many reads and writes to issue concurrently. A
typical simultaneous limit is 50 tracts, though the exact
value depends on the client’s bandwidth.

2.2

Deterministic data placement

A key issue in parallel storage systems is data placement and rendezvous, that is: how does a writer know
where to send data? How does a reader find data that has
been previously written?
Many systems solve this problem using a metadata
server that stores the location of data blocks [14, 30].
Writers contact the metadata server to find out where
to write a new block; the metadata server picks a data
server, durably stores that decision and returns it to the
writer. Readers contact the metadata server to find out
which servers store the extent to be read. This method
has the advantage of allowing maximum flexibility of
data placement and visibility into the system’s state.
However, it has drawbacks: the metadata server is a central point of failure, usually implemented as a replicated
state machine, that is on the critical path for all reads and
writes.
In FDS, we took a different approach. FDS uses a
metadata server, but its role during normal operations
is simple and limited: collect a list of the system’s active tractservers and distribute it to clients. We call this
list the tract locator table, or TLT. In a single-replicated
3
USENIX Association  

10th USENIX Symposium on Operating Systems Design and Implementation (OSDI ’12)  3


for a long time. Its size in a single-replicated system is
proportional to the number of tractservers in the system
(hundreds, or thousands), not the number of tracts stored
(millions or billions).
When the system is initialized, tractservers locally
store their position in the TLT. This means the metadata
server does not need to store durable state, simplifying its
implementation. In case of a metadata server failure, the
TLT is reconstructed by collecting the table assignments
from each tractserver.
To summarize, our metadata scheme has a number of
nice properties:

Newly created blobs have a length of 0 tracts. Applications must extend a blob before writing past the end
of it. The extend operation is atomic, is safe to execute
concurrently with other clients, and returns the new size
of the blob as a result of the client’s call. A separate API
tells the client the blob’s current size. Extend operations
for a blob are sent to the tractserver that owns that blob’s
metadata tract. The tractserver serializes it, atomically
updates the metadata, and returns the new size to each
caller. If all writers follow this pattern, the extend operation provides a range of tracts the caller may write without risk of conflict. Therefore, the extend API is functionally equivalent to the Google File System’s “atomic
append.” Space is allocated lazily on tractservers, so
tracts claimed but not used do not waste storage.

• The metadata server is in the critical path only when
a client process starts. This is the key factor that
allows us to practically keep tract sizes arbitrarily
small. Systems such as GFS [14] require larger
chunks partially to reduce load on the metadata
server.
• The TLT can be cached long-term since it changes
only on cluster configuration, not each read and
write, eliminating all traffic to the metadata server
in a running system under normal conditions.
• The metadata server stores metadata only about the
hardware configuration, not about blobs. Since traffic to it is low, its implementation is simple and
lightweight.
• Since the TLT contains random permutations of the
list of tractservers, sequential reads and writes by
independent clients are highly likely to utilize all
tractservers uniformly and are unlikely to organize
into synchronized convoys.

2.4

A result that flows naturally from FDS is that the assignment of work to worker can be done at very short
timescales. This enables FDS to mitigate stragglers—a
significant bottleneck in large systems because a task is
not complete until its slowest worker is complete [5].
Hadoop- and MapReduce-style clusters that primarily
process data locally are very sensitive to machines that
are slow due to factors such as misbehaving hardware,
jobs running concurrently, hotspots in the network, and
non-uniformity in the input. If a node falls behind, there
are not many options for recovery other than restarting
its computation elsewhere [5]. The straggler period can
also represent a great loss in efficiency if most resources
are idle while waiting for a slow task to complete.
In FDS, since storage and compute are no longer colocated, the assignment of work to worker can be done
dynamically, at fine granularity, during task execution.
The best practice for FDS applications is to centrally (or,
at large scale, hierarchically) give small units of work to
each worker as it nears completion of its previous unit.
This self-clocking system ensures that the maximum dispersion in completion times across the cluster is only the
time required for the slowest worker to complete a single
unit. Such a scheme is not practical in systems where the
assignment of work to workers is fixed in advance by the
requirement that data be resident at a particular worker
before the job begins.
In many applications, the effect is significant. For
example, in our sort application (§6.1), elimination of
stragglers in the reading phase accounted for a 1/3 reduction in total job runtime.

Our design is enabled by running on a full bisection
bandwidth network. The locality-oblivious uniform access pattern would cause crippling congestion on a traditional network with hierarchical oversubscription.

2.3

Dynamic Work Allocation

Per-Blob Metadata

Each blob has metadata such as its length. FDS stores
it in each blob’s special metadata tract (“tract −1”).
Clients find a blob’s metadata on a tractserver using
the same TLT used to find regular data. Distributed
metadata is a particular advantage for atomic blob operations that require serialization to avoid inconsistency
(e.g. CreateBlob, DeleteBlob and ExtendBlobSize).
Even if thousands of clients are requesting atomic operations on blobs simultaneously, operations that can be
parallelized (by virtue of referring to different blobs) are
likely serviced in parallel by independent tractservers.
When a blob is created, the tractserver responsible for
its metadata tract creates that tract on disk and initializes
the blob’s size to 0. When a blob is deleted, that tractserver deletes the metadata. A scrubber application scans
each tractserver for orphaned tracts with no associated
metadata, making them eligible for garbage collection.

3

Replication and Failure Recovery

Thus far, we have described FDS as single-replicated
and thus not resilient to disk failures. To improve durability and availability, FDS supports higher levels of
replication. When a disk fails, redundant copies of the
lost data are used to restore the data to full replication.
4

4  10th USENIX Symposium on Operating Systems Design and Implementation (OSDI ’12)  

USENIX Association


3.2

The use of full bisection bandwidth networks means
that, with appropriate data layout (§3.3), FDS can perform failure recovery dramatically faster than many other
systems. In an n-disk cluster where one disk fails,
roughly 1/nth of the replicated data will be found on
all n of the other disks. All remaining disks send the
under-replicated data to each other in parallel, restoring
the cluster to full replication very quickly. Performance
is bounded only by the aggregate disk and network bandwidth. Thus, as the size of the cluster grows failure recovery gets faster. As we will see in §5.3, our cluster
of about 1,000 disks recovers 92 GB of data lost from a
single disk in only 6.2 s, and 655 GB lost from 7 disks on
a failed machine in 33.7 s. Though the broad approach
of FDS’ failure recovery is similar to RAMCloud [26],
RAMCloud recovers data to DRAM and uses replication only for fault tolerance. FDS uses replication both
for availability and fault tolerance while recovering data
back to stable storage. Such fast failure recovery significantly improves durability because it reduces the window of vulnerability during which additional failures can
cause unrecoverable data loss.

3.1

Failure recovery

We begin with the simplest failure recovery case: the
failure of a single tractserver. Later sections will describe concurrent tractserver failures, support for failure
domains, and metadata server failures.
As described earlier, each row of the TLT lists several
tractservers. However, each row also has a version number, canonically assigned by the metadata server. When
a tractserver is assigned to a row and column of the TLT,
it is also given the row’s current version number.
Tractservers send heartbeat messages to the metadata
server. When the metadata server detects a tractserver
timeout, it declares the tractserver dead. Then, it:
• invalidates the current TLT by incrementing the version number of each row in which the failed tractserver appears;
• picks random tractservers to fill in the empty spaces
in the TLT where the dead tractserver appeared;
• sends updated TLT assignments to every server affected by the changes; and
• waits for each tractserver to ack the new TLT assignments, and then begins to give out the new TLT
to clients when queried for it.

Replication

As described in §2.2, each entry of the TLT in an nway replicated cluster contains n tractservers. (Construction of such a TLT is described in §3.3.) When an application writes a tract, the client library finds the appropriate row of the TLT and sends the write to every tractserver it contains. Reads select a single tractserver at
random. Applications are notified that their writes have
completed only after the client library receives write acknowledgments from all replicas.
Replication also requires changes to CreateBlob,
ExtendBlobSize, and DeleteBlob. Each mutates the
state of the metadata tract and must guarantee that updates are serialized. Clients send these operations only
to the tractserver acting as the primary replica, marked
as such in the TLT. When a tractserver receives one of
these operations, it executes a two-phase commit with the
other replicas. The primary replica does not commit the
change until all other replicas have completed successfully. Should the prepare fail, the operation is aborted.
FDS also supports per-blob variable replication, for
example, to single-replicate intermediate computations
for write performance, triple-replicate archival data for
durability, and over-replicate popular blobs to increase
read bandwidth. The maximum possible replication level
is determined when the cluster is created and drives the
number of tractservers listed in each TLT entry. Each
blob’s actual replication level is stored in the blob’s metadata tract and retrieved when a blob is opened. For an nway replicated blob, the client uses only the first n tractservers in each TLT entry.

When a tractserver receives an assignment of a new
entry in the TLT, it contacts the other replicas and begins
copying previously written tracts. When a failure occurs,
clients must wait only for the TLT to be updated; operations can continue while re-replication is still in progress.
All operations are tagged with the client’s TLT version. If a client attempts an operation using a stale TLT
entry, the tractserver detects the inconsistency and rejects
the operation. This prompts the client to retrieve an updated TLT from the metadata server. Client operations to
tractservers not affected by the failure proceed as usual.
Table versioning prevents a tractserver that failed and
then returned, e.g., due to a transient network outage,
from continuing to interact with applications as soon as
the client receives a new TLT. Further, any attempt by a
failed tractserver to contact the metadata server will result in the metadata server ordering its termination.
After a tractserver failure, the TLT immediately converges to provide applications the current location to read
or write data. This convergence property differentiates it
from other hash-based approaches, such as those used
within distributed hash tables, which may cause requests
to be routed multiple times through the network before
determining an up-to-date location for data.
3.2.1 Additional failure scenarios
Thus far, we have considered only the case where a
single tract server fails. We now extend our description
to concurrent and cascading tractserver failures as well
as metadata server failures.
5

USENIX Association  

10th USENIX Symposium on Operating Systems Design and Implementation (OSDI ’12)  5


Figure 2: A Tract Locator Table before (left) and after (right) Disk B fails. B’s appearances in the table are replaced
with different disks and the version numbers of affected rows are incremented. All the disks in rows that contained B
participate in failure recovery in parallel.
When multiple tractservers fail, the metadata server’s
only new task is to fill more slots in the TLT. Similarly, if
failure recovery is underway and additional tractservers
fail, the metadata server executes another round of the
protocol by filling in empty table entries and incrementing their version. Data loss occurs when all the tractservers within a row fail within the recovery window.

Tractservers that fail concurrently with a metadata
server are detected by the new metadata server as missing TLT entries; the normal failure recovery protocol is
executed. One strength our failure recovery protocol is
its simplicity; all tractserver failure cases use the same
protocol and exercise the same code path. This leads to
small, simple, more obviously-correct code.

Though not yet implemented, transient failures can be
handled gracefully with partial failure recovery. A tractserver failure triggers replication of lost data as usual.
If the tractserver later returns to service, the metadata
server has two options: complete failure recovery as if
the disk had never returned or use other replicas to recover the writes the returning tractserver missed while it
was away. The metadata server will choose the option
that requires copying less data. If it completes failure
recovery, the returning tractserver is added to the empty
disk pool. Otherwise, the tractserver resumes its positions in the TLT, failure recovery is halted, and copying of missed writes begins. Tractservers identify missed
writes by examining the version of the TLT with which
each tract was written. To further mitigate the effects of
transient faults, the metadata server could separate the
replacement of a failed server in the TLT from the initiation of the data movement required for failure recovery.

3.3

Replicated data layout

As mentioned previously, a k-way replicated system
has k tractservers (disks) listed in each TLT entry. The
selection of which k disks appear has an important impact
on both durability and recovery speed.
Imagine first that we wish to double-replicate (k = 2)
all data in a cluster with n disks. A simple TLT might
have n rows with each row listing disks i and i + 1. While
data will be double-replicated, the cluster will not meet
our goal of fast failure recovery: when a disk fails, its
backup data is stored on only two other disks (i + 1 and
i − 1). Recovery time will be limited by the bandwidth of
just two disks. A cluster with 1 TB disks with 100 MB/s
read performance would require 90 minutes for recovery.
A second failure within that time would have roughly a
2/n chance of losing data permanently.1
A better TLT has O(n2 ) entries. Each possible pair
of disks (ignoring failure domains; §3.3.1) appears in an
entry of the TLT. Since the generation of tract locators
is pseudo-random (§2.2), any data written to a disk will
have high diversity in the location of its replica. When
a disk fails, replicas of 1/nth of its data resides on the
other n disks in the cluster. When a disk fails, all n disks
can exchange data in parallel over FDS’ full bisection
bandwidth network. Since all disks recover in parallel,
larger clusters recover from disk loss more quickly.
While such a TLT recovers from single-disk failure
quickly, a second failure while recovery is in progress is

Network partitions complicate recovery from metadata server failures. A simple primary/backup scheme is
not safe because two active metadata servers will cause
cluster corruption. Our current solution is simple: only
one metadata server is allowed to execute at a time. If it
fails, an operator must ensure the old one is decommissioned and start a new one. We are experimenting with
using Paxos leader election to safely make this process
automatic and reduce the impact to availability. Once a
new metadata server starts, tractservers contact it with
their TLT assignments. After a timeout period, the new
metadata server reconstructs the old TLT.

1 More

2
precisely, 1 − ( n−1
n )

6
6  10th USENIX Symposium on Operating Systems Design and Implementation (OSDI ’12)  

USENIX Association


fined, FDS adheres to that policy when constructing the
tract locator table. FDS guarantees that none of the disks
in a single row of the TLT share the same failure domain. This policy is also followed during failure recovery: when a disk is replaced, the new disk must be in a
different failure domain than the other tractservers in that
particular row.

guaranteed to lose data. Since all pairs of disks appear as
TLT entries, any pair of failures will lose the tracts whose
TLT entry contained the pair of failed disks. Replicated
FDS clusters therefore have a minimum replication level
of 3. Perhaps counterintuitively, no level of replication
ever needs a TLT larger than O(n2 ). For any replication level k > 2, FDS starts with the “all-pairs” TLT, then
expands each entry with k − 2 additional random disks
(subject to failure domain constraints).
Constructing the replicated TLT this way has several
important properties. First, performance during recovery
still involves every disk in the cluster since every pair of
disks is still represented in the TLT.
Second, a triple disk failure within the recovery window now has only about a 2/n chance1 of causing permanent data loss. To understand why, imagine two disks
fail. Find the entries in the TLT that contain those two
disks. We expect to find 2 such entries. There is a 1/n
chance that a third disk failure will match the random
third disk in that TLT entry.
Finally, adding more replicas decreases the probability of data loss. Consider now a 4-way replicated cluster. Each entry in the O(n2 )-length TLT has two random
disks added instead of one. 3 or fewer simultaneous failures are safe; 4 simultaneous failures have a 1/n2 chance
of losing data. Similarly, 5-way replication means that
4 or fewer failures are safe and 5 simultaneous failures
have a 1/n3 chance of loss.
One possible disadvantage to a TLT with O(n2 ) entries
is its size. In our 1,000-disk cluster, the in-memory TLT
is about 13 MB. However, on larger clusters, quadratic
growth is cumbersome: 10,000 disks would require a
600 MB TLT.
We have two (unimplemented) strategies to mitigate
TLT size. First, a tractserver can manage multiple disks;
this reduces n by a factor of 5–10. Second, we can
limit the number of disks that participate in failure recovery. An O(n2 ) TLT uses every disk for recovery,
but 3,000 disks are expected to recover 1 TB in less than
20 s (§5.3). The marginal utility of involving more disks
may be small. To build an n-disk cluster where m disks
are involved in recovery, the TLT only needs O( mn × m2 )
entries. For 10,000 to 100,000 disks, this also reduces
table size by a factor of 5–10. Using both optimizations,
a 100,000 disk cluster’s TLT would be a few dozen MB.
3.3.1 Failure domains
A failure domain is a set of machines that have a high
probability of experiencing a correlated failure. Common failure domains include machines within a rack,
since they will often share a single power source, or machines within a container, as they may share common
cooling or power infrastructure.
FDS leaves it up to the administrator to define a failure domain policy for a cluster. Once that policy is de-

3.4

Cluster growth

FDS supports the dynamic growth of a cluster through
the addition of new disks and machines. For simplicity,
we first consider cluster growth in the absence of failures.
Cluster growth adds both storage capacity and
throughput to the system. The FDS metadata server rebalances the assignment of table entries so that both existing data and new workloads are uniformly distributed.
When a tractserver is added to the cluster, TLT entries are
taken away from existing tractservers and given to the
new server. These assignments happen in two phases.
First, the new tractserver is given the assignments but
they are marked as “pending” and the TLT version for
each entry is incremented. The new tractserver then begins copying data from other replicas. During this phase,
clients write data to both the existing servers and the new
server so that the new tractserver is kept up-to-date. Once
the tractserver has completed copying old data, the metadata server ‘commits’ the TLT entry by incrementing its
version and changing its assignment to the new tractserver. It also notifies the now replaced tractserver that it
can safely garbage collect all tracts associated with that
TLT entry.
If a new tractserver fails while its TLT entries are
pending, the metadata server increments the TLT entry
version and expunges it from the list of new tractservers.
If an existing server fails, the failure recovery protocol
executes. However, tractservers with pending TLT entries are not eligible to take over for failed servers as they
are already busy copying data.
Variable replication complicates cluster growth because each replica will have different data depending
upon the replication level of each blob written to it.
Therefore, new tractservers must read from the existing
tractserver whose TLT entry it is replacing. Should that
server fail, the new tractserver may read from another
tractserver “to the left” of its own entry, since it will have
a superset of the data required. For example, the first
replica in the TLT has all single-replicated tracts whereas
the third replica has all triple-replicated tracts.

3.5

Consistency guarantees

The current protocol for replication depends upon the
client to issue all writes to all replicas. This decision
means that FDS provides weak consistency guarantees
to clients. For example, if a client writes a tract to 1
of 3 replicas and then crashes, other clients reading dif7

USENIX Association  

10th USENIX Symposium on Operating Systems Design and Implementation (OSDI ’12)  7


ferent replicas of that tract will observe differing state.
Weak consistency guarantees are not uncommon; for example, clients of the Google File System [14] must handle garbage entries in files. However, if strong consistency guarantees are desired, FDS could be modified to
use chain replication [33] to provide strong consistency
guarantees for all updates to individual tracts.
Tractservers may also be inconsistent during failure
recovery. A tractserver recently assigned to a TLT entry will not have the same state as the entry’s other replicas until data copying is complete. While in this state,
tractservers reject read requests; clients use other replicas instead.

4

Each computer in our cluster has a dual-port 10 Gbps
NIC, primarily the Intel X520. One or both of these ports
are connected to a TOR depending on the server’s role
(compute vs. storage) and the number of data disks it
has. The NICs are configured with large-send offload,
receive-side scaling, and 9 kB (jumbo) Ethernet frames.
The TCP stack is configured with a reduced MinRTO to
quickly recover from loss.
We have found it difficult to saturate a 10 G NIC using
a single TCP flow: a single CPU core typically cannot
keep up with a 10 Gbps NIC’s interrupt load. Our operating system (Windows Server 2008 R2), in conjunction
with the Intel NIC, uses RSS (Receive Side Scaling) to
spread the interrupt load across cores. Similar to ECMP,
RSS prevents in-flow packet reordering by hashing the
TCP 4-tuple to select a core. As a result, multiple flows
are needed to spread interrupt load. We need 5 flows
per 10 Gbps port to reliably saturate the NIC. This is easily satisfied in FDS because its design dictates that many
flows are active simultaneously.
At 20 Gbps, a zero-copy architecture is mandatory.
FDS’ data interfaces pass the zero-copy model all the
way to the application. For clarity, Section 2.1 showed
conventional one-copy versions of WriteTract and
ReadTract interfaces; our applications actually use the
preferred zero-copy versions. We also use buffer pools
to avoid the large page fault penalty associated with frequent allocation of large buffers.

Networking

FDS’ main goal is to expose all of a cluster’s disk
bandwidth to applications. FDS creates an uncongested
path from disks to CPUs by:
• Giving each storage node network bandwidth equal
to its disk bandwidth, preventing bottlenecks between the disk and network;
• Using a full bisection bandwidth network, preventing bottlenecks in the network core; and
• Giving compute nodes as much network bandwidth
as they need I/O bandwidth, preventing bottlenecks
at the client
Our FDS testbed uses a two-layer CLOS network
[15, 16], which in its largest configuration consists of 8
“spine” routers and 14 “TORs” (Top-Of-Rack routers).
Each router is a 64×10 Gbps Blade G8264. Each TOR
has a 40 Gbps link (4 bonded 10 Gbps ports) to each spine
router, giving it 320 Gbps total bandwidth to the spine
layer. The other 320 Gbps of each TOR’s bandwidth attaches to NICs. In total, this provides about 5.5 Tbps of
bisection bandwidth for an infrastructure cost of about
$250k. The routers are factory-standard, running the
manufacturer’s OS (BladeOS v6.8.4). We use BGP for
route distribution with each TOR on its own IP subnet.
The TORs load-balance traffic to the spine using
ECMP (equal-cost multipath routing), a standard router
feature that selects a spine route for each TCP flow based
on the hash of the TCP destination. This gives the
network full bisection bandwidth without the need for
global scheduling, and has an important advantage over
round-robin route selection: it ensures packets within a
flow are not reordered.
One drawback of ECMP is that full bisection bandwidth is not guaranteed, but only stochastically likely
across multiple flows. Long-lived, high-bandwidth flows
are known to be problematic with ECMP [3]. FDS, however, was designed to use a large number of short-lived
(tract-sized) flows to a large, pseudo-random sequence of
destinations. This was done in part to satisfy the stochastic requirement of ECMP.

4.1

RTS/CTS

By design, at peak load, all FDS nodes simultaneously
saturate their NICs with short, bursty flows. A disadvantage of short flows is that TCP’s bandwidth allocation algorithms perform poorly. Under the high degree of fan-in
seen during reads, high packet loss can occur as queues
fill during bursts. The reaction of standard TCP to such
losses can have a devastating effect on performance. This
is sometimes called incast [34].
Schemes such as DCTCP [4] ameliorate incast in
concert with routers’ explicit congestion notification
(ECN). However, because our network has full bisection bandwidth, collisions mostly occur at the receiver,
not in the network core, providing us an opportunity
to prevent them in the application layer. FDS does
so with a request-to-send/clear-to-send (RTS/CTS) flowscheduling system. Large sends are queued at the sender,
and the receiver is notified with an RTS. The receiver
limits the number of CTSs outstanding, thus limiting the
number of senders competing for its receive bandwidth
and colliding at its switch port.
RTS/CTS adds an RTT to each large message. This
has the potential to reduce performance by preventing
the network pipeline from ever filling. However, the FDS
API encourages deep read-ahead and write-ahead, ensur8

8  10th USENIX Symposium on Operating Systems Design and Implementation (OSDI ’12)  

USENIX Association


Bandwidth (MB/sec)

130
120
110
100
90
80
70
60
50
40
30
20
10
0

of 24 10,000 RPM SAS disks reading continuously using system calls ranging from 8KB to 32MB. Reads are
sequential on the upper line, random on the lower line.
Sequential performance peaks at about 131 MB/s. 8 MB
random reads reach ≈117 MB/s, or ≈88% of sequential
performance. Write performance, not shown, is similar
to read performance.

Sequential

Random

5.2
10

50

SimpleTestClient is the “hello world” FDS application.
It uses the FDS client library to read and write blobs from
and to tractservers. As in a real application, data are sent
over the network and read from or written to disk. We
tested its throughput and scalability by running successively larger numbers of SimpleTestClient instances concurrently against 1,033 tractservers. Each client instance
had a single 10G NIC assigned to it. The tract size was
8 MB. For each configuration we adjusted the blob size
so that the total read or write lasted between 5 and 10
minutes. We plot the number of clients against their aggregate throughput averaged over the entire experiment.
Figure 4a shows 1 to 180 clients reading and writing
blobs sequentially against an unreplicated cluster. Two
pairs of curves are plotted: a 516-disk and 1,033-disk
cluster. At the left of the graph, the total tractserver bandwidth far exceeds the client bandwidth; performance is
constrained by the number of clients. Throughput increases linearly at the rate of about 1,150 MB/s/client for
writing and 950 MB/s/client for reading, roughly 90%
and 74%, respectively, of the 10 Gbps interface. Reading tends to be slower than writing due to the difficulty
of maintaining NIC saturation under fan-in (§4.1).
The right of the curves flatten as client bandwidth
grows larger than tractserver bandwidth, which saturates them. Performance peaks at 32 GB/s in the 516disk configuration and 67 GB/s in the 1,033-disk configuration, showing near-linear scalability. In both cases,
FDS achieves remote reading and writing at roughly half
the locally achievable disk throughput measured in §5.1.
Though much better than many existing blob storage systems (see Table 2) there is room for improvement.
A similar test using random reads and writes is shown
in Figure 4b. As expected, the performance is substantially the same as the sequential read and write tests. This
validates our design goal that for 8MB tracts, random and
sequential I/O deliver the same performance.
Figure 4c shows the bandwidth of sequentially reading
and writing clients against a 1,033 disk triple-replicated
cluster. As expected, the write bandwidth is about onethird of the read bandwidth since clients must send three
copies of each write. Scaling properties are similar to
that seen in Figures 4a and 4b.
Finally, to test the maximum speed achievable by a
single client process, we tested a single instance of Sim-

500
5000
Read Size (KB)

Figure 3: Performance of a single process reading to a
single 10,000 RPM disk. Each point is the mean across
24 disks. Error bars show the standard deviation.
ing the FDS network library always has a large queue of
future messages. This allows it to send an RTS for future
messages in parallel with other data transfers, allowing
the pipeline to fill.
Small and large messages are delivered using separate
TCP flows, reducing the latency of control messages by
enabling them to bypass long queues. FDS network message sizes are bimodal: large messages are almost all
about 8 MB, and most others are a few kB or less.

5

Microbenchmarks

In this section and §6, we describe microbenchmarks
and application benchmarks. All of our tests were conducted on a heterogeneous cluster of up to 256 HP, Dell
and Silicon Mechanics servers acquired between 2008
and 2012. The machines had between 12 and 96 GB
RAM, between 2 and 24 cores, and between 0 and 20
data disks (plus one OS disk). The disks were a combination of 147 GB and 300 GB 10,000 RPM 2.5” dual-port
SAS drives; and 500 GB and 1 TB 7,200 RPM 3.5” SATA
drives. Dynamic work allocation (§2.4) was key to efficiently using such a heterogeneous cluster. Since some
machines were on loan to us during our quest to break the
sort record (§6.1), some experiments used a subset of the
cluster. In all but one (§6.3) of our experiments, computation and storage are located on different machines,
which guarantees all experiments rely exclusively on remote access to storage. Unless otherwise noted, each test
used an unreplicated cluster with CRC checks disabled.
The network configuration is described in §4.

5.1

Remote Reading and Writing

Raw Disk Performance

We start with a simple test: a single process writing to
a single local disk. This benchmark establishes the baseline performance of our cluster’s disks and shows how
large random reads must be to amortize the cost of disk
seeks. Like the tractserver, this benchmark uses the raw
disk interface (that is, does not use NTFS). Results are
shown in Figure 3. Each point is the median performance
9
USENIX Association  

10th USENIX Symposium on Operating Systems Design and Implementation (OSDI ’12)  9


Total Bandwidth (GB/s)

70
60
50
40
30
20
10
0

read
write

1,033 Disks

516 Disks

1

2

5 10
50
Number of Clients

70
60
50
40
30
20
10
0

200

70
60
50
40
30
20
10
0

read
write

1

2

5 10
50
Number of Clients

200

read
write

1

2

5 10
50
Number of Clients

200

Figure 4: Mean aggregate throughput of 1 to 180 clients reading and writing 8 MB tracts on a 1,033-disk cluster.
Standard deviation is less than 1% of the mean of each point. The x axes use logarithmic scales. (a) Sequential reading
and writing in a single-replicated cluster. Results for a 516-disk cluster are also shown. (b) Random reading and
writing in a single-replicated cluster. (c) Sequential reading and writing in a triple-replicated cluster.
Disk count
Disks failed
Total (TB)
GB/disk
GB recov.
Recovery
time (s)

100
1
4.7
47
47
19.2
±0.7

1
9.2
92
92
50.5
±16.0

1
47
47
47
3.3
±0.6

1,000
1
92
92
92
6.2
±0.4

data. FDS recovered the lost data in 33.7 ± 1.5 s. Compared to single-disk failures, the whole-machine failure
test recovered 7× the data in only 5× the time, as the
fixed costs of recovery were amortized over more linear
recovery time.
Recovery of 655 GB involves reading and writing a
total of 1310 GB. Doing so in 33.7 s with 993 tractservers implies an average total read/write bandwidth of
≈40 MB/s/disk. Since the disks in these tests were nearly
full, recovery wrote to the innermost, slowest disk tracks.
These results imply that a 1 TB disk in a 3,000 disk
cluster could be recovered in ≈17 s. Such a small recovery window dramatically lessens the probability of data
loss. Further, it lowers the impact of disk failures on
availability and application performance. Though we are
recovering to disk, our recovery time is comparable to
the best known technique for recovering to RAM [26].

7
92
92
655
33.7
±1.5

Table 1: Mean and standard deviation of recovery time
after disk failure in a triple-replicated cluster. The high
variance in one experiment is due to a single 80 s run.
pleTestClient with 20 Gbps assigned to it, rather than
10 Gbps as in the previous tests. We wrote, then read,
10,000 tracts against a single-replicated cluster of 30
tractservers. Over 5 trials, SimpleTestClient achieved
a mean (and standard deviation) write bandwidth of
2,187 ± 32 MB/s; for read, 2,045 ± 15 MB/s.

5.3

6
6.1

Failure Recovery

Table 1 shows the time taken to re-replicate lost data
after one or more disk failures. Each experiment used
a triple-replicated cluster of 100 or 1,000 146 GB, 10 K
RPM SAS disks that contained enough data so that each
tractserver held either 47 GB or 92 GB. We then killed a
random tractserver process and measured the time until
the cluster reported failure recovery was complete. Due
to the random nature of the TLT, the exact amount of data
recovered in each test varied slightly. We ran each test 5
times and we report the mean and standard deviation.
With 100 disks, FDS recovered 47 GB in 19.2 s and
92 GB in 50.5 s. Scaling the number of disks up by 10×
to 1,000, recovery times improved by 6.8× to 3.3 s for
100 disks and by 8× to 6.2 s for 1,000 disks. Scaling is
not quite linear due to the fixed cost of generating and
distributing a new TLT.
We measured whole-machine failure by resetting a
machine running 7 tractservers containing ≈655 GB of

Applications
Sort

Sorting is an important primitive in many big-data applications. Its load pattern is similar to other common
tasks such as distributed database joins and large matrix
operations. This has made it an important benchmark
since at least 1985 [9]. A group informally sponsored
by SIGMOD curates an annual disk-to-disk sort performance competition with divisions for speed, cost efficiency, and energy efficiency [2]. Each has sub-divisions
for general purpose applications (“Daytona”) and implementations that are allowed to exploit the specifics of the
competition (“Indy”), such as assuming 100-byte records
and uniformly distributed 10-byte keys.
In April 2012, our FDS-based sort application set the
world record for sort in both the Indy and Daytona categories of MinuteSort [7], which measures the amount
of data that can be sorted in 60 seconds. Using a cluster
of 1,033 disks and 256 computers (136 for tractservers,
120 for the application), our Daytona-class app sorted
10

10  10th USENIX Symposium on Operating Systems Design and Implementation (OSDI ’12) 

USENIX Association


System

Computers

Data Disks

Sort Size

Time

Implied Disk
Throughput

MinuteSort—Daytona class (general purpose)
FDS, 2012
256
1,033 1,401 GB
59 s
Yahoo!, Hadoop, 2009 [25]
1,408
5,632
500 GB
59 s
Yahoo!, Hadoop, 2009 [25]
1,408
5,632 1,000 GB
62 s
(unofficial 1 TB run)
MinuteSort—Indy class (benchmark-specific optimizations allowed)
FDS, 2012
256
1,033 1,470 GB 59.4 s
UCSD TritonSort, 2011 [27]
66
1,056 1,353 GB 59.2 s

46 MB/s
3 MB/s
5.7 MB/s

47.9 MB/s
43.3 MB/s

800GB Sort Before Dynamic Work Allocation
100
90 Output complete
80
70
Shuffled data received from all peers
60
50
40
30
20
10 Input read complete
0
0
20
40
60
80
100

Milestone Completion Time (s)

Milestone Completion Time (s)

Table 2: Comparison of FDS MinuteSort results with the previously standing records. In accordance with sort benchmark rules, all reported times are the median of 15 runs and 1 GB = 109 bytes.

120

Number of Nodes Complete

800GB Sort After Dynamic Work Allocation
100
90
80
70
60 Output complete
50
40 Shuffled data received from all peers
30
20 Input read complete
10
0
0
20
40
60
80
100

120

Number of Nodes Complete

Figure 5: Visualization of the time to reach three milestones in the completion of a sort. The results are shown before
(left) and after (right) implementation of dynamic work allocation. Both experiments depict 115 nodes sorting 800 GB.
1,401 GB2 in 59.4 s. This bested the standing Daytona
record by a factor of 2.8x while using about 1/5 as many
CPUs and disks. Our Indy-class app sorted 1,470 GB in
59.0 s, breaking the standing record by about 8%. Our
Indy sort was identical to Daytona except for assuming
a uniform key distribution instead of sampling the input. Our results and comparisons to the previous recordholders are shown in Table 2. The sort application consists of one head process and n worker processes. The
input is given in a single FDS blob, from which each
worker reads a separate subset of tracts. The sort occurs
in three phases: input sampling, reading, and writing.
In the input sampling phase, the head process reads
1.5 million records: 6,000 from each of 256 tracts selected randomly from the input blob. It computes the key
distribution and hence the assignment of key ranges to
buckets. It then unicasts the computed distribution to all
the other sort processes. In the Indy sort, this phase is
skipped; the key distribution is assumed to be uniform
and the bucket partitions are pre-computed.
In the read phase, each sort process performs three
tasks simultaneously. First, each sort process reads tracts

from its assigned region of the input. Simultaneously,
as each tract arrives (in arbitrary order), the sort process
shuffles the tract’s records into output bins according to
the bucket partitions received after the sampling phase.
As each bin fills, the process sends the bin to the appropriate “bucket-receiver,” a peer sort process. The buckets
form an ordered partition of the keyspace. Finally, it receives bins from peer sort nodes.
In the write phase, each sort process sorts its bucket
and writes it to a separate output blob. The sorted result
is distributed among n blobs.
Sort relied heavily on dynamic work allocation (§2.4).
At first, all workers were responsible for reading equal
portions of the input; 1/3 of the total sort time was lost to
stragglers, partially because of our cluster’s heterogeneity. In later versions, the head node coordinated work assignments: nodes requested an input range as they neared
completion of their previously assigned range.
Figure 5 shows time diagrams for a sort experiment
before and after dynamic work allocation was implemented. (No other changes were made between the two
experiments shown.) Each graph has three lines that depict the time at which each worker reached three milestones: completing the read phase, starting the write

2 In this section only, we use the sort benchmark’s definition of 1 GB
= 109 bytes.

11
USENIX Association  

10th USENIX Symposium on Operating Systems Design and Implementation (OSDI ’12)  11


phase, and completing the write phase. Horizontal lines
would be ideal, indicating every node reached a milestone simultaneously. There is a global barrier between
the read and write phases; stragglers in the read phase
cause significant performance loss as most nodes wait
idly for the last readers to complete. Dynamic work allocation, enabled by FDS’ freeing nodes from data locality,
significantly improved overall sort time by reducing the
time nodes were waiting at the barrier.
FDS is the first system in the history of the sort benchmark to set the record using general-purpose remote storage exclusively, without exploiting locality at all. Parallel sorts were previously accomplished by reading input
data off a local disk, bucketing over the network, then
writing to a local disk. In FDS, all storage is remote: our
sort reads data from remote tractservers, buckets, then
writes back to tractservers, sending data over the network
three times. This demonstrates that FDS achieves worldclass performance while still exposing a simple singledisk model that frees applications from the complexity
associated with reasoning about locality.
In 2011, TritonSort [27] set the MinuteSort record
with code carefully optimized for sorting. The FDS
sort achieved similar per-disk bandwidth despite being
built on top of a general blob store, accessing disks remotely. However, the FDS result did use approximately
4× as many computers. Part of this difference is superficial: tractservers were run on older machines, recycled
from other projects; they did not have sufficient CPU
or memory to act as sort clients. A single modern machine could act as both tractserver and client whereas our
cluster required two. However, some of the difference is
fundamental; it demonstrates the inefficiency of records
traversing the network three times rather than once.

6.2

System
Public cloud
FDS

Cores
1
8

Time
(sec)
1,200
24

Speedup
Per Node
1
50x

Speedup
Per Core
1
6.25x

Table 3: Comparison of FDS to a public cloud for reading a single day of stock market data and generating one
order book per stock symbol found.
A 6× speedup might actually under-state the improvement from FDS. On the cloud computing system, the
VM’s single core was underutilized. A multi-core VM
would have been unlikely to significantly improve the
speed since the bottleneck was I/O.
An interesting performance bottleneck emerged. Because this task had always been I/O-bound, the input
was stored after zlib compression and decompressed on
the fly. This effectively increased I/O throughput by
the compression factor of 7×. In FDS, input tracts arrived so quickly that decompression was the bottleneck.
8 MB compressed tracts were, on average, arriving every
70 ms per 1 Gbps NIC. Zlib decompression took 218 ms,
implying 3 cores were needed per NIC, but the machine had only 8 cores. Configuring zlib to favor speed
rather than compression reduced decompression time to
only 188 ms. Switching to a compression library optimized for decompression speed (XPress), compression
ratio was reduced to 3:1 but tract decompression required
only 62 ms. FDS transformed cointegration from an I/Obound task to a compute-bound task.

6.3

Serving an index of the web

Search portals such as Bing use an index of the web to
provide answers to search queries. In 2011, Bing was
evaluating alternative architectures for serving the tail
(i.e., infrequent) queries from the index. We ported one
system to use FDS and measured its performance.
The tail index serving pipeline is composed, roughly,
of two stages. First, for each term found in a user’s
search query, a list of the documents that contain it are
retrieved from disk. The document list comes annotated
with a static rank of each document’s relevance with respect to that single search term. Second, these document
lists are merged, ranked, and sent to a secondary ranker
which computes each document’s relevance to the total
query based on document contents and other information. These document lists were “term-sharded,” meaning each machine in the cluster was responsible for a
subset of the terms found in all crawled web documents.
Each machine spread its document lists across four local
disks. The code had been carefully optimized.
Our effort to use FDS focused initially on a feasibility
experiment that perturbed as little as possible in the index
serving pipeline. Each rack of machines was converted

Cointegration

A colleague had an application that performs cointegration (a statistical technique) on stock market data to
find correlations in price fluctuations. The raw input is
a time-series of all stock trades. The application’s first
phase reads the time-series data and, for each ticker symbol it finds, generates an “order book,” a list of trades for
that stock on that day. The second phase compares pairs
of order books to find correlations.
This application was originally implemented on a publicly available cloud-computing provider. Measurements
showed it was I/O-bound, so he ported it to FDS. The
public cloud implementation used one single-core VM
for compute and the cloud service’s blob store for storage. The FDS version used one 8-core machine for compute and 98 tractservers in an older version of our cluster
that used 9 1 Gbps Ethernet interfaces per machine. Accounting for differences in hardware, the FDS version
was at least 6× faster, as shown in Table 3.
12

12  10th USENIX Symposium on Operating Systems Design and Implementation (OSDI ’12) 

USENIX Association


data into the cluster using metadata tracts, which further
minimizes the need to centrally administer the store.
xFS [6] also proposed distributing file system metadata among storage nodes through distributed metadata
managers. A replicated manager map determined which
manager was responsible for the location of a particular
file. FDS distributes its metadata among storage nodes
through the metadata tract and uses deterministic placement to eliminate the need for an additional manager service to respond to requests for the location of individual
files within the storage system.
DHTs like Chord [31] use techniques such as consistent hashing [20] to eliminating the need for centralized
coordination when locating data. However, under churn,
a request within a DHT might be routed to several different servers before finding an up-to-date location for data.
FDS also uses hashing to implement deterministic placement, but the TLT directs clients to data without ambiguity. Failures spur the generation of a new table, providing
an up-to-date location to clients. FDS takes advantage of
deterministic placement to minimize the load on the FDS
metadata server while relying on a small amount of state
to ensure that the location of data is determined within a
single network hop.
RAMCloud [26] provides fast recovery from failures
by distributing data, in the form of log segments, across
many machines and recovering those segments in parallel. However, RAMCloud recovers all data to main memory, while FDS implements fast failure recovery back
to stable storage. Further, RAMCloud distributes data
to disk purely for reasons of fault tolerance, while FDS
replication is used both for fault tolerance and availability. Panasas [35] uses RAID 5 to stripe files, rather than
blocks, across many servers to accelerate RAID recovery, while FDS ensures that all disks participate in recovery by striping tracts among all disks in the cluster.
Distributed file systems like Frangiapani [32],
GPFS [28], AFS [18] and NFS [29] export a remote
storage model across a shared, hierarchical name-space.
These systems must contend with strong consistency
guarantees, and the vagaries of remote, shared access to
a POSIX compliant API. By focusing only on fast access
to blob storage, FDS provides weak consistency guarantees with very high performance.
Among many others, systems such as Swift [11], Zebra [17], GPFS [28], Panasas [35], and Petal [21] stripe
files, blocks, or logs across file servers to improve read
and write throughput for traditional hierarchical file systems. FDS follows in the footsteps of these systems by
using the tract locator table to guarantee a uniform distribution of disk accesses regardless of the pattern of requests issued by applications.
TritonSort [27] demonstrated the power of a balanced
approach to storage and computation by breaking several

into an FDS cluster. The index serving pipeline, also
running on the same machines, was changed to use FDS
as its “local” store instead of local physical disks. This
had the effect of striping all shards’ data across all disks
in the rack.
We ran in-house performance regression tests that issued thousands of queries to the ranker pipeline and measured the number of queries per second (QPS) served
while keeping 95% of service times below the maximum
specified by the SLA. We measured performance using
two configurations. The first consisted of an unmodified
index serving pipeline on 40 machines, each of which
had 4 disks and a 10 Gb/s network connection to a single
switch. The second configuration used only 12 machines
but ran using an FDS cluster with 48 tractservers.
The FDS version showed a dramatic QPS improvement of 2× despite using 1/3 the machines. While the
median latency was essentially the same in both versions,
the FDS version reduced the 95th percentile latency by
2.4×. We attribute the difference to better statistical multiplexing of disks. Queries require document lists that
vary in size by several orders of magnitude. Using local
disks, long reads delay other queries behind them. FDS,
in contrast, was better at spreading document lists more
uniformly across all available disks, making it less likely
that single machine would fall behind.
The index serving pipeline proved to be a good match
for our weak consistency model. While serving the index, the document lists are read-only. When the index is
updated, each term’s blob is written by a single writer,
and the front-end does not try to read the new blob until
the update is complete.

7

Related Work

We believe that FDS is the first high performance blob
storage system designed for datacenter scale with a flat
storage model.
The Google File System has a centralized master that
keeps all metadata in memory [14]. This approach is
limiting because as the contents of the store grow, the
metadata server becomes a centralized scaling and performance bottleneck. In a recent interview, Google architects described the GFS metadata server as a limiting
factor in terms of scale and performance [22]. Additionally, a desire to reduce the size of a chunk from 64 MB
was limited by the proportional increase in the number
of chunks in the system.
FDS uses deterministic placement to eliminate the
scaling limitations of current blob store metadata servers.
The tract locator table’s size is determined by the number of machines in a cluster, rather than the size of its
contents. Further, the TLT enables FDS to use an unlimited number of small, 8 MB tracts to stripe data across
the cluster. Finally, FDS fully distributes the blob meta13
USENIX Association  

10th USENIX Symposium on Operating Systems Design and Implementation (OSDI ’12)  13


9

of the world records in sorting. FDS demonstrates that
this approach can be extended to a general purpose, locality oblivious system to break the sorting record without loss of performance.
PortLand [24] and VL2 [15] make it economically feasible to build datacenter-scale full bisection bandwidth
networks. Other full bisection networks exist, such as
Infiniband [23], but at a cost and scale that limited them
to supercomputing and HPC environments.
Finally, River [8] used a distributed queue to dynamically adjust the assignment of work to applications nodes
at run time in data flow computations. Similarly, FDS applications use dynamic work allocation to choose which
node will consume a tract of data at runtime to adjust for
performance faults and the varying performance capabilities within a heterogeneous cluster.

8

Acknowledgments

We thank our shepherd, Steve Gribble, and the anonymous reviewers for helpful comments and feedback.
John Douceur, Jason Flinn and Eddie Kohler also provided insightful comments on early drafts. We are indebted to Barry Bond, who ported the cointegration application to FDS and has provided extensive guidance on
performance-tuning Windows Server. Dave Maltz built
our first CLOS networks and taught us how to build our
own. Johnson Apacible, Rich Draves, and Reuben Olinsky were part of the sort record team. Trevor Eberl, Jamie
Lee, Oleg Losinets and Lucas Williamson provided systems support. Galen Hunt provided a continuous stream
of optimism and general encouragement. Li Lu helped to
integrate FDS with the Bing index serving pipeline. We
also thank Jim Larus for agreeing to fund our initial 14machine cluster on nothing more than a whiteboard and
a promise, allowing this work to flourish.

Conclusion

References

We have presented Flat Datacenter Storage, a
datacenter-scale blob storage system that exposes the full
bandwidth of its disks to all processors uniformly. It
largely obviates the need for locality without sacrificing
performance. Individual processes can read and write at
near their NIC’s rate—2 GBps or more. Aggregate client
bandwidth scales nearly linearly and reaches 50% of the
theoretical bandwidth of the underlying disks. Bandwidth capacity scales nearly linearly as disks are added.
This has a number of important consequences. First,
recovery from failed disks can be done in seconds rather
than hours; we recover 655 GB in a 1,000 disk cluster in
only 33.5 s. Small recovery windows increase durability
by decreasing the likelihood of complete data loss.
Second, FDS has implications for the structure of software. By exposing a cluster’s full I/O bandwidth without locality constraints, FDS deconflates high I/O performance from data-parallel programming models such
as MapReduce. Programmers can pick the most natural model for expressing computation without sacrificing
performance.
Third, FDS has implications for the way clusters are
built. Today’s big-data clusters are often built with “one
size fits all” machines that assume all applications have a
similar balance of CPU to disk bandwidth requirements.
With FDS, I/O and compute resources can be purchased
separately, each independently upgradable depending on
which resource is in shortage.
Finally, systems like FDS may pave the way for new
kinds of applications. Large matrix operations, sorts,
distributed joins, and all-to-all comparisons were largely
off-limits to programmers working at datacenter scales:
if it couldn’t be done within a rack, it couldn’t be
done quickly. FDS makes these applications practical—
potentially even enabling new kinds of science.

[1] Apache Hadoop. http://hadoop.apache.org.
[2] MinuteSort Benchmark. http://sortbenchmark.org.
[3] M. Al-Fares, S. Radhakrishnan, B. Raghavan, N. Huang, and
A. Vahdat. Hedera: Dynamic flow scheduling for data center networks. In the 8th USENIX Symposium on Networked Systems Design and Implementation (NSDI ’10), pages 281–296. USENIX
Association, 2010.
[4] M. Alizadeh, A. G. Greenberg, D. A. Maltz, J. Padhye, P. Patel,
B. Prabhakar, S. Sengupta, and M. Sridharan. Data center TCP
(DCTCP). In S. Kalyanaraman, V. N. Padmanabhan, K. K. Ramakrishnan, R. Shorey, and G. M. Voelker, editors, SIGCOMM,
pages 63–74. ACM, 2010.
[5] G. Ananthanarayanan, S. Kandula, A. Greenberg, I. Stoica, Y. Lu,
B. Saha, and E. Harris. Reining in the outliers in map-reduce clusters using mantri. In the 9th USENIX Symposium on Operating
Systems Design and Implementation (OSDI ’09), October 2010.
[6] T. E. Anderson, M. D. Dahlin, J. M. Neefe, D. A. Patterson, D. S.
Roselli, and R. Y. Wang. Serverless network file systems. ACM
Trans. Comput. Syst., 14(1):41–79, Feb. 1996.
[7] J. Apacible, R. Draves, J. Elson, J. Fan, O. Hofmann, J. Howell, E. Nightingale, R. Olinksy, and Y. Suzue. MinuteSort with
Flat Datacenter Storage. Technical report, Microsoft Research,
http://sortbenchmark.org/FlatDatacenterStorage2012.pdf, 2012.
[8] R. H. Arpaci-Dusseau. Run-time adaptation in River. ACM Trans.
Comput. Syst., 21(1):36–86, Feb. 2003.
[9] D. Bitton, M. Brown, R. Catell, S. Ceri, T. Chou, D. DeWitt,
D. Gawlick, H. Garcia-Molina, B. Good, J. Gray, P. Homan,
B. Jolls, T. Lukes, E. Lazowska, J. Nauman, M. Pong, A. Spector, K. Trieber, H. Sammer, O. Serlin, M. Stonebraker, A. Reuter,
and P. Weinberger. A measure of transaction processing power.
Datamation, 31(7):112–118, Apr. 1985.
[10] D. Borthakur. The Amazon Simple Storage Service (Amazon
S3). http://aws.amazon.com/s3/.
[11] L.-F. Cabrera and D. D. Long. Swift: Using distributed disk striping to provide high I/O data rates. Computing Systems, 4(4):405–
436, 1991.
[12] Cisco Systems. Data center: Load balancing Data Center Services, 2004.

14
14  10th USENIX Symposium on Operating Systems Design and Implementation (OSDI ’12) 

USENIX Association


[13] J. Dean and S. Ghemawat. MapReduce: Simplified data processing on large clusters. In The 6th Symposium on Operating
Systems Design and Implementation (OSDI ’04), pages 137–150,
December 2004.

[25] O. O’Malley and A. C. Murthy. Winning a 60 second dash with
a yellow elephant. http://sortbenchmark.org/Yahoo2009.
pdf, 2009.
[26] D. Ongaro, S. M. Rumble, R. Stutsman, J. Ousterhout, and
M. Rosenblum. Fast crash recovery in RAMCloud. In Proceedings of the Twenty-Third ACM Symposium on Operating Systems
Principles (SOSP ’11), pages 29–41, New York, NY, USA, 2011.
ACM.

[14] S. Ghemawat, H. Gobioff, and S.-T. Leung. The Google file system. In Proceedings of the Nineteenth ACM Symposium on Operating Systems Principles (SOSP ’03), pages 29–43, New York,
NY, USA, 2003. ACM.
[15] A. Greenberg, J. R. Hamilton, N. Jain, S. Kandula, C. Kim,
P. Lahiri, D. A. Maltz, P. Patel, and S. Sengupta. VL2: a scalable
and flexible data center network. Commun. ACM, 54(3):95–104,
Mar. 2011.

[27] A. Rasmussen, G. Porter, M. Conley, H. M. andRadhika Niranjan Mysore, A. Pucher, and A. Vahdat. Tritonsort: A balanced
large-scale sorting system. In the 8th USENIX Symposium on Networked Systems Design and Implementation (NSDI ’11), Boston,
MA, April 2011.

[16] A. Greenberg, P. Lahiri, D. A. Maltz, P. Patel, and S. Sengupta.
Towards a next generation data center architecture: scalability
and commoditization. In Proceedings of the ACM workshop
on Programmable Routers for Extensible Services of Tomorrow,
PRESTO ’08, pages 57–62, New York, NY, USA, 2008. ACM.

[28] F. Schmuck and R. Haskin. GPFS: A shared-disk file system for
large computing clusters. In the Conference on File and Storage
Technologies (FAST ’02), January 2002.
[29] S. Shepler, B. Callaghan, D. Robinson, R. Thurlow, C. Beame,
M. Eisler, and D. Noveck. Network File System (NFS) version 4
Protocol. RFC 3530, Apr. 2003.

[17] J. H. Hartman and J. K. Ousterhout. The Zebra striped network
file system. In Proceedings of the Fourteenth ACM Symposium
on Operating Systems Principles (SOSP ’93), pages 29–43, New
York, NY, USA, 1993. ACM.

[30] K. Shvachko, H. Kuang, S. Radia, and R. Chansler. The hadoop
distributed file system. In Mass Storage Systems and Technologies (MSST), 2010 IEEE 26th Symposium on, pages 1–10, May
2010.

[18] J. H. Howard, M. L. Kazar, S. G. Menees, D. A. Nichols,
M. Satyanarayanan, R. N. Sidebotham, and M. J. West. Scale
and performance in a distributed file system. ACM Transactions
on Computer Systems, 6(1), February 1988.

[31] I. Stoica, R. Morris, D. Karger, M. F. Kaashoek, and H. Balakrishnan. Chord: A scalable peer-to-peer lookup service for internet
applications. In Proceedings of the 2001 Conference on Applications, Technologies, Architectures, and Protocols for Computer
Communications, SIGCOMM ’01, pages 149–160, New York,
NY, USA, 2001. ACM.

[19] M. Isard, M. Budiu, Y. Yu, A. Birrell, and D. Fetterly. Dryad: Distributed data-parallel programs from sequential building blocks.
In Proceedings of the 2007 Eurosys Conference, pages 59–72,
2007.
[20] D. Karger, A. Sherman, A. Berkheimer, B. Bogstad, R. Dhanidina, K. Iwamoto, B. Kim, L. Matkins, and Y. Yerushalmi. Web
caching with consistent hashing. Computer Networks, 31:1203–
1213, May 1999.

[32] C. A. Thekkath, T. Mann, and E. K. Lee. Frangipani: a scalable distributed file system. In Proceedings of the Sixteenth ACM
Symposium on Operating Systems Principles (SOSP ’97), pages
224–237, New York, NY, USA, 1997. ACM.

[21] E. K. Lee and C. A. Thekkath. Petal: distributed virtual disks. In
ASPLOS-VII: Proceedings of the Seventh International Conference on Architectural support for Programming Languages and
Operating Systems, volume 31, pages 84–92, New York, NY,
USA, Sept. 1996. ACM.

[33] R. van Renesse and F. B. Schneider. Chain replication for supporting high throughput and availability. In The 6th Symposium
on Operating Systems Design and Implementation (OSDI ’04),
pages 91–104, 2004.
[34] V. Vasudevan, H. Shah, A. Phanishayee, E. Krevat, D. Andersen,
G. Ganger, and G. Gibson. Solving TCP incast in cluster storage
systems. In The 7th USENIX Conference on File and Storage
Technologies (FAST ’09), San Francisco, CA, February 2009.

[22] M. K. McKusick and S. Quinlan. GFS: Evolution on fastforward. acmqueue, 7(7), August 2009.
[23] Mellanox. Building a Scalable Storage with InfiniBand, 2012.
[24] R. Niranjan Mysore, A. Pamboris, N. Farrington, N. Huang,
P. Miri, S. Radhakrishnan, V. Subramanya, and A. Vahdat. Portland: a scalable fault-tolerant layer 2 data center network fabric.
In Proceedings of the ACM SIGCOMM 2009 conference on Data
communication, SIGCOMM ’09, pages 39–50, New York, NY,
USA, 2009. ACM.

[35] B. Welch, M. Unangst, Z. Abbasi, G. Gibson, B. Mueller,
J. Small1, J. Zelenka, and B. Zhou. Scalable performance of the
Panasas parallel file system. In The 6th USENIX Conference on
File and Storage Technologies (FAST ’08), 2008.

15
USENIX Association  

10th USENIX Symposium on Operating Systems Design and Implementation (OSDI ’12)  15


